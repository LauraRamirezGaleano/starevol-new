\documentclass[10pt]{article}
\usepackage{colordvi,psfig,subfigure,amssymb,picinpar,ulem,longtable}
\usepackage[dvips]{graphics,color}
\usepackage{makeidx}
\usepackage[a4paper]{hyperref}
\usepackage{soul}

\makeindex
\textwidth 200mm
\hoffset -42mm
\textheight 275mm
\voffset -30mm

\def\dive{{\rm div}}
\def\ga{\mathrel{\mathchoice {\vcenter{\offinterlineskip\halign{\hfil
$\displaystyle##$\hfil\cr>\cr\sim\cr}}}
{\vcenter{\offinterlineskip\halign{\hfil$\textstyle##$\hfil\cr>\cr\sim\cr}}}
{\vcenter{\offinterlineskip\halign{\hfil$\scriptstyle##$\hfil\cr>\cr\sim\cr}}}
{\vcenter{\offinterlineskip\halign{\hfil$\scriptscriptstyle##$\hfil
\cr>\cr\sim\cr}}}}}

\def\la{\mathrel{\mathchoice {\vcenter{\offinterlineskip\halign{\hfil
$\displaystyle##$\hfil\cr<\cr\sim\cr}}}
{\vcenter{\offinterlineskip\halign{\hfil$\textstyle##$\hfil\cr<\cr\sim\cr}}}
{\vcenter{\offinterlineskip\halign{\hfil$\scriptstyle##$\hfil\cr<\cr\sim\cr}}}
{\vcenter{\offinterlineskip\halign{\hfil$\scriptscriptstyle##$\hfil
\cr<\cr\sim\cr}}}}}


%\newcommand{\WPREF}[1]{\hyperlink{#1}{\texttt{#1}\index{\texttt{#1}|hyperpage}}}
%\newcommand{\XREF}[2]{\WPREF{#2} (\texttt{#1}\index{#1|hyperpage})}

\newcommand{\HPTO}[1]{\hypertarget{#1}{\sout{\texttt{#1}} \index{\texttt{#1}}}}
\newcommand{\HPT}[1]{\hypertarget{#1}{\texttt{#1} \index{\texttt{#1}}}}
\newcommand{\HPL}[1]{\hyperlink{#1}{\texttt{#1}}}
\newcommand{\LA}[1]{\textcolor{red}{#1}}

\begin{document}


\begin{center}
{\Large \sf \textbf{PARAMETER CARD}}

{\sf starevol4.00 - Jan 2023}
\end{center}

\vspace*{-0.5cm}

\begin{center}
\begin{longtable}{|p {2.2cm}|p {2.35cm}|p {12.5cm}|}\hline
\texttt{variable} & \textbf{values} & meaning\\  \hline\hline

\multicolumn{3}{|c|}{\sf evolutionary sequences} \\ \hline\hline
\HPT{maxmod} & 0 $\rightarrow  n $  &
Number of models in the current sequence, (0) stops the calculation. \\ 
\hline
{\HPT{imodpr}} & 11, 12, 33
& Activate flame routines (12 no printings), (33) save more models during.
pulse/DUP  \\ \hline 
\HPT{mtinit} & $M_\odot$ & Initial stellar mass in solar masses.
\\ \hline 
\HPT{zkint} & $Z_{\mathrm{tab}}$ & Initial metallicity. \\ \hline\hline
\multicolumn{3}{|c|}{\sf equation of state} \\ \hline\hline
\HPT{addH2} & f, t & Inclusion of $\mathrm{H}_{2}$.
\\ \hline 
\HPT{addHm} & f, t & Inclusion of $\mathrm{H}^{-}$.
\\ \hline 
\HPT{Tmaxioh} & $\sim 1 \times 10^5$ &
Temperature above which H is considered completely ionized (if ionizHe = f).
\\ \hline 
\HPT{Tmaxiohe} & $\sim 5 \times 10^5$ &  Temperature
above which He is considered completely ionized (if ionizHe = f). \\ \hline
\HPT{ionizHe} & f, t & Inclusion of He partial
ionization (should put t if atomic diffusion active). \\ \hline 
\HPT{ionizC} & f, t & Inclusion of C partial
ionization (should put t if atomic diffusion active). \\ \hline 
\HPT{ionizN} & f, t & Inclusion of N partial
ionization (should put t if atomic diffusion active). \\ \hline 
\HPT{ionizO} & f, t & Inclusion of O partial
ionization (should put t if atomic diffusion active). \\ \hline 
\HPT{ionizNe} & f, t & Inclusion of Ne partial
ionization (should put t if atomic diffusion active). \\ \hline\hline 

\multicolumn{3}{|c|}{\sf gravitation} \\ \hline\hline
\HPT{lgrav} & 1, 2, 4 & 
\begin{tabular}{l}
\hspace{-0.25cm}Different prescriptions for $\varepsilon_{\rm grav}$. \\ 
1 : $\varepsilon_{\rm grav} = - D e_{\rm int}/D t - P D(1/\rho)/Dt$ \\
2 : $\varepsilon_{\rm grav} = -c_P D T/D t + Q/\rho \, (D P/D t)$ \\
4 : $\varepsilon_{\rm grav} = - D e_{\rm int}/D t - 4 \pi P\, \partial (r^2 V )/\partial m_r$ \\
\end{tabular}
\\ \hline
\HPT{lnucl} & t, f &  $\varepsilon_{\rm nuc}$
calculated at each iteration (t) or only once the model has converged (f).\\
\hline\hline 

\multicolumn{3}{|c|}{\sf convection} \\ \hline\hline
\HPT{alphac} & $\alpha _{c}$ & 
$\alpha _{c} = \Lambda /H_{P}$ classical MLT free parameter with the pressure scale height (obtained from solar calibration, check "How to Starevol" document). \\
\hline
{\HPT{dtnmix}} & f, t, g, u & 
\begin{tabular}{ll}
f : & instantaneous mixing, {\sf lmix NOT activated} \\
t : & time-dependent convective mixing {\sf lmix NOT activated} \\ 
g : & idem f but {\sf \HPT{lmix} activated} \\
u : & idem t but {\sf lmix activated} \\  \hline
\end{tabular} \\ \hline
\HPT{mixopt} & f, t & Convective zone homogenized at each iteration
during the convergence (t) 
or only when the structure has converged (f). \\ \hline
{\HPT{hpmlt}} & 1 $\rightarrow$ 6 & 
\begin{tabular}{ll}
\multicolumn{2}{l}{\hspace{-0.3cm} Different MLT formalisms with the pressure scale height.} \\
1 &: Cox \& Guili formalism (sometimes more stable; first sequence) \\
3 &: Kippenhahn formalism (usually used) \\
2 $\rightarrow$ 4 &: idem (1 $\rightarrow$ 3) but using time dependent MLT \\
5 &: Convection model with compression effects \\
  &\hspace{0.2cm} Forestini, Lumer, Arnould (1991,A\&A,252,127) \\
6 &: Rotationally modified convection following \\
  &\hspace{0.2cm} Augustson \& Mathis (2019) \\
  &\hspace{0.2cm} Currently not used !
\end{tabular}
\\ \hline
{\HPT{nczm}} & $-9 \rightarrow 99$ & Merge (remove) convective
zones that are separated by less (which extent is less) than \HPL{nczm}
shells. If \HPL{nczm} $ <0$ the procedure applies only in the envelope.
\\ \hline
\HPT{ihro} & f, t & MLT with $H_{P}$ (f) or $H_{\rho }$ (t) prescription.
\\ \hline
\HPT{\sout{iturb}} & 0, 1 & 
\begin{tabular}{l}
Inclusion (1) or not (0) of a turbulent flux in the convective energy balance. \\ 
(with the $H_{\rho }$ MLT prescription only) \\
Currently not used !
\end{tabular}
\\ \hline
{\HPT{etaturb}} & $\eta _{\rm turb}$ & 
\begin{tabular}{l}
Parameter to compute the turbulent flux of the convective cells. \\ 
(with the $H_{\rho}$ MLT prescription only) \\
According to Maeder (1987), \HPL{etaturb} should be of the order of 1000.
\end{tabular}
\\ \hline
{\HPT{fover}} & $f_{\rm over}$ & 
\begin{tabular}{l}
With \HPL{lover}$\in [23,32]$, corresponds to Herwig's overshoot parameter $f_{\rm over}$. \\
With \HPL{lover}$\in [33,39]$ or $\in [70-73]$, corresponds to the penetration depth $d_{\rm ov}$ \\
for Baraffe 2017, Augustson \& Mathis 2019, Korre 2019. \\
(0.0325: Li Cluster PMS)
\end{tabular}
\\ \hline

\newpage
\hline \textbf{variable} & \textbf{values} & meaning \\ \hline\hline \hline

\HPT{alphatu} & $\alpha _{\rm turb}$ & $\alpha _{\rm turb} = \Lambda /H_{\rho}$ classical MLT free parameter with the density scale height or free parameter for convective cells with $H_{P}$ and hpmlt=5 (check "How to Starevol" document for $\alpha_{\rm MLT}$ values).\\ \hline
\HPT{novopt} & 0, 1, 2 & 
\begin{tabular}{l}
Treatment of the step overshooting (if $\neq 0$); radial upward extension
limited by \\ 
$\alpha _{\rm over,up}H_{P}$ (1) or the minimum of $\alpha _{\rm over,up}H_{P}$ and $%
\alpha _{\rm over,up}r$ (2).
\end{tabular}
\\ \hline
\HPT{aup} & $\alpha _{\rm over,up}$ & Radial downward extension in case of step
overshooting, 
characterizes core overshooting. \\ \hline
\HPT{adwn} & $\alpha _{\rm over,down}$ & Radial upward extension in case of step
overshooting, 
characterizes envelope undershooting.  \\ \hline
{\HPT{idup}} & 0, 1, 2, 3, 5 & (1) find neutrality of the
gradients (AGB phase only), (2) call to convzone stopped after
 iterations, (3) : (1) + (2), (5) : overshoot below envelope (\LA{\HPL{lover} = 23})
activated during AGB phase only. \\ \hline

\multicolumn{3}{|c|}{\sf atmosphere} \\ \hline\hline
\HPT{tau0} & $\tau _{0} \la 0.1$ & Fixed optical
depth of the last numerical shell (to set when starting a model). \\ \hline
{\HPT{ntprof}} & 0, 1 $\rightarrow$ 8 &
\begin{tabular}{ll}
\multicolumn{2}{l}{\hspace{-0.3cm} Boundary condition :}  \\
(0) & grey atmosphere  \\
(1) & analytic fit \\
(2) & no atmosphere \\
(3) & \LA{Siess et al. 2000 fit} \\
(4) & \LA{PHOENIX realistic atmosphere model} \\
(5) & \LA{Krishna-Swami 1966 fit} \\ 
(6) & CMFGEN atmosphere models (massive stars) \\
    & Currently not used ! \\
(7) & TLUSTY atmosphere models (massive stars) \\
    & Currently not used ! \\
(8) & Vernazza et al. (1981) analytic fit as quoted in Spnoi et al. (2019) \\
    & Currently not used !
\end{tabular}
\\ \hline 
\HPT{taulim} & $> \tau _{0}$ & Optical depth
above which the Eddington (grey) approximation is used (to set when starting a model).\\ \hline 
\multicolumn{3}{|c|}{\sf mass loss} \\ \hline\hline
\HPT{mlp} & $0,$ $1\rightarrow 30$ & 
\begin{tabular}{ll}
\multicolumn{2}{l}{\hspace{-0.3cm} Mass loss prescription (if $\neq 0$)}. \\
 1,2   & : Reimers (1975) : RGB\\
 3,4   & : de Jager (1988 A\&AS, 72, 259) for log(Teff)$>$3.7, \\
       & \hspace{0.2cm} Crowther (2001) for log(Teff)$<$3.7. \\
       & \hspace{0.2cm} Massive stars \\
 5,6   & : Vassiliadis \& Wood (1993, ApJ 413, 641) : AGB, no delay of the onset \\
       & \hspace{0.2cm} of the super-wind phase\\
 55,56   & : Vassiliadis \& Wood (1993, ApJ 413, 641) : AGB, original prescription\\
 7,8   & : Blocker (1995, A\&A 297, 727) : AGB\\
 9,10  & : Arndt (1997, A\&A 327, 614) : AGB \\
 11,12 & : Schaller et al. (1992) : Massive stars and WR phase \\
 13,14 & : Chiosi (1981, A\&A 93, 163) : Massive O stars \\
 15,16 & : Vink et al. (2001) for log(Teff)$>$4.0969, \\
       & \hspace{0.2cm} de Jager (1988,A\&A,72,259) for 4.0969$>$log(Teff)$>$3.7, \\
       & \hspace{0.2cm} Crowther (2001) for log(Teff)$<$3.7, \\
       & \hspace{0.2cm} Nugis \& Lamers (2000) for $H_{\rm surf}<$0.4 and log(Teff)$>$4.0. \\
       & \hspace{0.2cm} Massive stars, MS, post-MS, RSG and WR phase \\
 17,18 & : user defined, equal to \HPL{massrate} \LA{(Accretion)} \\
 19,20 & : Crowther (2000) : for RSG \\
 21,22 & : Van Loon et al. (2005) : for AGB and RSG \\
 23,24 & : Cranmer \& Saar (2011) : for cool PMS (23 only), MS and RGB stars \\
       & \hspace{0.2cm} (if rotation and a convective envelope) \\
 25,26 & : Graefener (2021) : VMS at LMC metallicity \\
 27,28 & : Sanders \& Vink (2022) : VMS stars \\
 29,30 & : Sabhahit et al. (2023) : VMS stars at low Z \\
\end{tabular}
 If mlp is odd, mass is removed in shells above m=0.95*mtot.
 If mlp is even, shells are removed.
\\ \hline

\newpage
\hline \textbf{variable} & \textbf{values} & meaning \\ \hline\hline \hline

\HPT{etapar}$^\dag$ & $\eta _{\mathrm{Reimers}}$ & Free parameter of the
Reimers (1975) mass loss prescription. \\ \hline
\HPT{dmlinc}$^\dag$ & $>1$ or $<1$ & Factor by which the mass loss rate is 
artificially multiplied. \\ \hline
\HPT{clumpfac} & $<$ 1 & Factor by which the Vink et al. (2001) mass loss is artificially multiplied to take into account the effects of clumping. \\ \hline
\HPT{zscaling} & f, t & If t, multiply the mass loss rate by a factor $(Z/Z_\odot)^{0.8}$, where $Z_\odot$ is the solar metallicity. \\ \hline\hline

\multicolumn{3}{|c|}{\sf nuclear} \\ \hline\hline
{\HPT{nucreg}} &
$0\rightarrow 3$ & 
\begin{tabular}{l}
\hspace{-0.25cm}Treatment of the nucleosynthesis.  \\ 
0 : no nucleosynthesis (nuclear energy production $\varepsilon_{\rm nuc}$
calculated) \\  
1 : nucleosynthesis calculated after convergence \\
2 : idem 1 but convective zones treated radiatively \\
3 : nucleosynthesis calculated at each iteration during the convergence
process  
\end{tabular}
\\ \hline
{\HPT{nuclopt}} & \begin{tabular}{l}\hspace{-0.25cm}i,j,h,n,m,c,l,\\\hspace{-0.25cm}u,p,f\end{tabular} & 
\begin{tabular}{l}
i : mix unstable nuclei inside convective zones even if $\tau_{\rm decay}
\ll \tau_{\rm conv}$ \\
j : i + deactivate special treatment of Li, Be and B during HBB \\ (see diffusion.f) \\
h : deactivate processes linked to HBB \\
n : compute neutron equilibrium abundance after nucleosynthesis \\
m : homogenize neutron abundances in the convective zones \\
c : authorize coupled resolution of mixing and nucleosynthesis if the
luminosity \\variation is $>$ \HPL{ftnuc} and the star is not undergoing
a TP (for p injection)\\
l : if decoupling hypothesis not satisfied (see \HPL{ftnuc}), model
is recomputed \\with a smaller timestep else time step is not allowed to
increase \\
u : active URCA terms (Eloc and Econv) \\
p : if \HPL{diffzc} = f, convection zone can be treated as
  radiative (\HPT{partialmix} on) \\
f : do none of the above
\end{tabular}
\\ \hline
\HPT{tolnuc} & $>0$ & 
\hspace{-0.25cm}\begin{tabular}{l}
Tolerance on mass fraction conservation in the nuclear subroutine \\ 
tolerance = $10^{-10} \times$ \HPL{tolnuc}.
\end{tabular}
\\ \hline
{\HPT{ftnuc}$^\dag$} & $< 1 $ & Check that the nucleosynthesis
does not modify too much the energetics. \\
& &  Maximum authorized change in
the nuclear luminosity :
$\Bigl|1-\frac{L_{\rm nuc}(X^{n+1})}{L_{\rm nuc}(X^n)}\Bigr| < $
\HPL{ftnuc}\\ \hline 
\HPT{Znetmax} & $>0$ & 
Nuclear charge (Z) of the last considered species (only for large networks).
\\ \hline\hline

\multicolumn{3}{|c|}{\sf diffusion} \\ \hline\hline
\HPT{idiffcc} & f, t & Computation of slow
particle transport processes (diffusion). \\ \hline
{\HPT{diffnuc}} & 1,
       {2}, 3  &  
\hspace{-0.25cm}\begin{tabular}{l}
Computation of slow particle transport processes (if $\neq 0$, default :
2), \\  
before (1), after (2) or before and after (3) the nucleosynthesis.
\end{tabular}
\\ \hline
{\HPT{idiffty}} & $4,8 \rightarrow 15$ & 
\begin{tabular}{l}
\hspace{-0.3cm} Rotational mixing. \\
4   : mixing recipe used in Charbonnel ApJ 1995 \\
8   : chemical mixing + AM transport Talon 1997 \\
9   : AM transport only, Talon 1997 \\
10  : (8)+(\HPL{lmicro} = 2) \\
11  : (\HPL{lmicro} = 3) + chemical mixing +  AM transport Maeder,Zahn 1998 \\
13  : (11) + non stationary terms in for AM transport (include everything) \\
14  : $\Omega$ evolves as a results of structural changes, evolution with
$j_k  = cste$ \\
15  : (13) + additional viscosity \HPT{nu$\_$add} \\
\end{tabular}
\\ \hline
\HPT{diffzc} & f, t & Compute convective mixing as a diffusive process
(implies radiative nucleosynthesis).\\ \hline
\HPT{diffst} & $>1$ & Factor by which the evolution timestep is reduced
to compute the diffusion. \\ \hline
\HPT{zgradmu} & f, t & Computation of angular momentum
  and chemical transports including the feedback on molecular weight when
\HPL{idiffty} = 8,9 or 11. \\ \hline
\HPT{nu$\_$add} & $\nu_{\rm add}$ & Additional viscosity in cm$^2$/s (\HPL{idiffty}=15). \\ \hline
\HPT{del$\_$zc} & $< 1$ & Minimum value allowed for $\nabla_{\rm RAD}$. \\ \hline
{\HPT{ledouxmlt}} & f,t & Use Ledoux criterion. In the
semiconvective zone mixing is treated diffusively. \\ \hline
{\HPT{Tfix}} & $T_{\rm fix}$ & Fixation point of the turbulence function of temperature. \\ \hline

\newpage
\hline \textbf{variable} & \textbf{values} & meaning \\ \hline\hline \hline

{\HPT{om\_turbul}} & $\Omega_{\rm turbul}$ & Coefficient entering the expression of the turbulent diffusion coefficient for chemicals according to Richard et al. 2005, ApJ 619, 538, Equation 2. Multiplicative factor compared to the He atomic diffusion
coefficient.
Dturbul is proportional to om\_turbul*DHe0. \\ \hline
{\HPT{n\_turbul}} & $n_{\rm turbul}$ & Exponent indicating the density dependence of the turbulent diffusion coefficient for chemicals according to Richard et al. 2005, ApJ 619, 538, Equation 2. Dturbul is proportional to $1/\rho^{\rm n\_turbul}$. \\ \hline
{\HPT{PM\_turbul}} & $PM_{\rm turbul}$ & Coefficient entering the turbulent diffusion coefficient expression according to Proffitt \& Michaud (1991a) as expressed in  Richard et al. 2005, ApJ 619, 538, Equation 3. Dturbul2 is proportional to PM\_turbul*$\rho^{-3}$. \\ \hline

\hline\multicolumn{3}{|c|}{\sf rotation} \\ \hline\hline
{\HPT{omegaconv}} & f,t,s,m & Treatment of AM transport in convective zone
(f=exclude convective core and envelope ONLY - t=include ALL
conv. zones, s = solid-body rotation in the entire star (routine {\sf
  rot\_sol} - m = no AM transport but chemical mixing). \\ \hline

{\HPT{Dh\_prescr}} & name & \begin{tabular}{l}
\hspace{-0.3cm} Prescription for horizontal diffusion coefficient Dh. \\
Zahn1992 : Zahn 1992 \\
Maeder03 : Maeder 2003 \\
MPZ\_2004 : Mathis, Palacios \& Zahn 2004 \\
Maeder06 : Maeder 2006 \\
Mathis16 : Mathis 2017, $\tau = 1/S$ \\
Mathis02 : Mathis 2017, $\tau = 1/(2\Omega + S)$ \\
Mathiepi : Mathis 2017, $\tau = 1/N_\Omega$
\end{tabular}
\\ \hline
\HPT{dm$\_$ext} & $> 1$ & Parameter used to smooth the diffusion
  coefficient at the limit of the convective core. \\ \hline
\HPT{Dv\_prescr} & name & \begin{tabular}{l}
\hspace{-0.3cm} Prescription for vertical diffusion coefficient Dv. \\
Za92 : Zahn 1992 \\
TZ97 : Talon \& Zahn 1997 (do not use with \HPL{Dh\_prescr}=Mathis16) \\
Ma97 : Maeder 1997 \\
Pr16 : Mathis et al. 2018 (TB updated - not fully functional yet) \\
\end{tabular}
\\ \hline
\HPT{om$\_$sat} & 08-15 & Saturation value for the dynamo-generated magnetic field
(in solar angular velocity). \\ \hline
\HPT{D$\_$zc} & &  Diffusion coefficient in the convective zone. \\ \hline
{\HPT{disctime}} & & Defines the duration of the
  disk-locking phase during the pre-main sequence. Expressed in years. (usually 2.5$\times10^6$ for rapid rotation and 5$\times10^6$ for median and slow rotation) \\
  \hline
\HPT{\begin{tabular}{l}\hspace{-0.25cm}thermal$\_$\\\hspace{-0.25cm}equilibrium \end{tabular}} & f,t & Starting from thermal equilibrium in the
 computation of angular momentum transport. \\ \hline
\HPT{diffvr} & ${\rm V}_{\rm init}$ & Initial velocity in km.${\rm
    s}^{-1}$ (if no magnetic braking). \\ \hline
\HPT{idiffvr} & 0 $\rightarrow$ 9 & \begin{tabular}{ll}
\hline\multicolumn{2}{c}{\hspace{-0.3cm} Prescription for braking at the stellar surface.} \\
(0) & no torque applied \\
(1) & constant rotation speed\\
(2) & power law for breaking~: $\Omega = \Omega_0
  (t/t_0)^n$ \\
(3) & Skumanich breaking law~: $\Omega =
(p\, \Omega_0^{-1} + \rm cte)^{-1/(p-1)}$ \\
(4) & constant angular velocity\\
(5) & torque applied $\alpha$ ${\Omega_S}^3$ \\
(6) & constant specific angular momentum \\
(7) & gfdsf \\
(8) & Matt et al. 2012 (with \HPL{mlp} = 23,24)\\
(9) & Matt et al. 2015 \\
\end{tabular}
\\ \hline
\HPT{idiffex} &  & Exponent intervening in
the breaking law (if \HPL{idiffvr}=2,3).\\ \hline
\HPT{diffcst} & & Constant intervening in the different braking laws. Examples : (a) 4e47 with \HPL{idiffvr} = 5 for a solid body rotating 1M$_\odot$ model, (b) $\simeq$ 2e48 for a differentially rotating model. (c) $\simeq$ 3e30 with \HPL{idiffvr} = 9. (Hardcoded for \HPL{idiffvr} = 8). (corresponds to the value of K*2/3 if Matt 2015 is used, solar case : 5$\times$10$^{30}$erg)\\ \hline
\HPT{breaktime} & & \begin{tabular}{l}
If idiffvr = 3 date in years from which the breaking of the star starts \\ when Skumanich braking law
is used.  \\
If idiffvr $\ne$ 3 corresponds to the angular velocity of the disk (and thus the \\ stellar surface) during the disc-locking phase on the PMS. Expressed in s$^{-1}$.\\
\end{tabular} 
\\ \hline

\newpage
\hline \textbf{variable} & \textbf{values} & meaning \\ \hline\hline \hline

\multicolumn{3}{|c|}{\sf time-dependent convection}
\\ \hline \hline
{\HPT{nmixd}} & 0, $1
\rightarrow 5$ &  
\begin{tabular}{l}
\hspace{-0.25cm}Computation of time-dependent mixing (tdm) and
nucleosynthesis (if $\neq 0$).
\\ (1)  core only \\(2) envelope only \\ (3) core \& envelope \\ (4) all
convective zones \\ (5) through all the star
\end{tabular}
\\ \hline
\HPT{itmind} & $>1$ & Minimum number of
iterations for tdm computations. \\ \hline
\HPT{itmaxd} & $<99$ & Maximum number of
iterations for tdm computations. \\ \hline
\HPT{rprecd} & $\epsilon _{\rm diff}$ & Precision to be reached on the abundances for the solution to converge. \\
\hline {\HPT{nretry}} & various &
\begin{tabular}{ll}
\multicolumn{2}{l}{\hspace{-0.3cm}Parameter used in different contexts according to the
selected values.} \\
 1     & : used in the call of the opacity routine kappa to
         include the \\
       & \hspace{0.2cm} computation of molecular opacities
         (= opamol in kappa.f)\\
 4     & : used in structure to redefine thermodynamic quantities at convective \\
       & \hspace{0.2cm} boundaries in case of rotating models where the effective gravity \\
       & \hspace{0.2cm} accounts for centrifugal forces \\
 other & : number of shells in the shock neighbourhood \\
       & \hspace{0.2cm} (in case of Super AGB stars) \\
\end{tabular}\\ \hline \hline

\multicolumn{3}{|c|}{\sf extra mixing} \\ \hline\hline
{\HPT{lover}}  & 23 - 71 & 
\begin{tabular}{ll}
\multicolumn{2}{l}{\hspace{-0.3cm} Overshoot prescription.} \\
23    & : below convective envelope \\
24    & : above core\\
25    & : below pulse\\
26    & : above pulse\\
27    & : above core and below envelope (=23+24)\\
28    & : below and above pulse (=25+26)\\
29    & : treat all cases (=27+28)\\
30    & : overshoot below all convective zones\\
31    & : overshoot above all convective zones\\
32    & : overshoot everywhere (=30+31)\\
33    & : overshoot Baraffe 2017 \\
34-36 & : overshoot dependent of rotation Augustson \& Mathis 2019 below the \\
      & \hspace{0.2cm} base of the CZ (34), above the base of the CZ (35), or both (36) \\
37-39 & : overshoot dependent of rotation Korre 2019 below the base of the CZ \\
      & \hspace{0.2cm} (37), above the base of the CZ \\
      & \hspace{0.2cm} (38), or both (39) \\
41-43 & : IGW from the envelope only (41), from the core (42), \\
      & \hspace{0.2cm} or from both (43) \\
60,61 & : parametric turbulence Richard 2005 set at a temperature threshold \\
      & \hspace{0.2cm} (60), or at the base of the CZ (61) \\
70    & : 34 + 60 \\
71    & : 34 + 61 \\
\end{tabular}
\\ \hline
\HPT{lthal} & 0 - 1 &  Thermohaline mixing (0 = off, 1 = on). \\ \hline
{\HPT{ltach}} & 41 - 44 &
\begin{tabular}{l}
\hspace{-0.3cm} Tachocline mixing. \\
41 : independent of time Brun 1999 \\
43 : dependent of time Brun 1999
\end{tabular} \\ \hline
\HPT{lmicro} & 2 - 6 & 
\begin{tabular}{l}
\hspace{-0.3cm} Atomic diffusion. \\
2 : Chapman \& Cowling \\
3 : Montmerle \& Michaud 1976, Paquette 1986, partial ionization \\
4 : Thoul 1994, Paquette 1986, partial ionization \\
5 : Montmerle \& Michaud 1976, Paquette 1986, total ionization \\
6 : Thoul 1994, Paquette 1986, total ionization \\
\end{tabular}
\\ \hline
\multicolumn{3}{|c|}{\sf accretion} \\ \hline\hline
\HPT{iaccr} & $0$, $1\rightarrow 4$ & 
\begin{tabular}{l}
\hspace{-0.25cm}Computation of the accreted matter profile (if $\neq 0$) \\ 
without (1, 2) or with (3, 4) shear energy production rate $\varepsilon
_{\rm shr}$ \\ 
with two different normalizations : \\
1 or 3 : $\rm M_{acc} = \int \frac{facc}{1+facc}\ d m$ \\
2 or 4  : $\rm M_{acc} = \int facc \ d m$  (reference)
\smallskip
\end{tabular}
\\ \hline

\newpage
\hline \textbf{variable} & \textbf{values} & meaning \\ \hline\hline \hline

\HPT{accphase} & $0$, $1\rightarrow 8$ & 
\begin{tabular}{l}
\hspace{-0.25cm}Different computations of the accreted matter profile. \\ 
if $0$ : accretion model, including D burning (suited for PMS phase) \\
if $[1,4]$ :  accretion inside the star (planet accretion)\\
if $5$ : uniform accretion from the surface with $\rm facc =
$\HPL{ric} (\HPL{menv} unknown) \\ 
if $6$ :  uniform accretion from the surface $M_\star$ to \HPL{menv} ($\rm
facc$ unknown)\\
if $7$ :  uniform accretion from the surface $M_\star$  to $M_\star
-\HPL{menv}$  ($\rm facc$ unknown)\\
if $8$ :  uniform accretion from the surface $M_\star$  to
\HPL{menv}$\times M_\star$  ($\rm facc$ unknown)\\
\end{tabular}
\\ \hline
\HPT{itacc} & $0\rightarrow 4$ &
\begin{tabular}{l} If \HPL{accphase} = 0, 
 different prescriptions for $^2$H mixing in
case of accretion. \\ 
If \HPL{itacc} = 0, matter pills up at the surface of the star. \\
If \HPL{itacc} $>$ 0, matter mixing with the surface layers. \\
\end{tabular}
\\ \hline
\HPT{massrate} & $\ge 0$ & Mass accretion rate in M$_\odot$\,yr$^{-1}$.\\ \hline
\HPT{massend} & $M_{\star ,\max }$ & Accretion is stopped when $M >$ \HPL{massend}. \\ \hline
\HPT{menv} & $M_{\star ,\max }$ & Mass is deposited from \HPL{menv}
to the surface (works with \HPL{accphase} = 6). \\ \hline
\HPT{ric} & $\mathcal{R}i_{\rm conv}$ & Richardson number for convective
regions. \\ \hline
\HPT{prrc} & $\mathcal{P}e$ & Peclet number characterizing the thermal
behavior of the accreted matter. \\ \hline
\HPT{xiaccr} & $0 \le \xi \le 1$ & Angular momentum
fraction actually accreted inside the star. \\ \hline
\HPT{fdtacc} & $\ge 1$ & 
Factor by which the evolution timestep is modified
in case of non-convergence of the accretion procedure (case
\HPL{accphase}=0).
\\ \hline
\HPT{alphaccr} & $0 \le \alpha_{\rm acc} \le 1 $ & Fraction of the
accretion luminosity deposited in the star. \\ \hline\hline
\multicolumn{3}{|c|}{\sf hydrodynamics} \\ \hline\hline
{\HPT{hydrodynamics}} & 0, 1, 2, 3  & Hydrodynamics off (0) on (1), (2) and (3)
idem but with rotational forces included. \\
\hline
\HPT{ivisc} & 0, 1, 2 &
Prescriptions for the artificial viscosity.\ \ 
\hspace{-0.25cm}\begin{tabular}{l}
0 : no artificial viscosity \\
1 : $P_{\rm visc} = q_0 \, l^2 \rho \,(D\ln \rho/D t)^2 $\\
2 : $P_{\rm visc} = q_0 \, l^2 \rho^3 \,(4 \pi \, \partial (r^2 v)/\partial
m_r)^2$ \\
3 : $P_{\rm visc} = q_0 \, l^2 \rho \,|\dive\,v| (dv/dr-1/3\,\dive\,v)$
\end{tabular}
\\ \hline
\HPT{q0}$^\dag$ & $q_0 \simeq 1-2$ & Parameter of the artificial viscosity. \\
\hline
\HPT{mcut} & $0 \le m_{\rm cut} \le M_{\star}$ & Mass cut above which
artificial viscosity is activated. \\
\hline\hline

\multicolumn{3}{|c|}{\sf shell masses}\\\hline\hline
\HPT{maxsh} & $0\, ...\, n$ & Maximum number of shells that can be changed by
the mesh laws. \\ \hline
{\HPT{nresconv}} & $ -9 \rightarrow 99$ & 
\begin{tabular}{l}
Spatial resolution increased in at least {\HPL{nresconv}} shells
around a convective \\
boundary - if $\HPL{nresconv} <0$ zonetest {NOT} activated {\sf else} zonetest
activated \\
(better to put at 0 if atomic diffusion active). 
\end{tabular}
\\ \hline 
\HPT{dlnvma} & $\displaystyle \Bigl\{\frac{\Delta X}{X}\Bigl\}_{\max} < 1$ & 
\hspace{-0.25cm}\begin{tabular}{l} Spatial resolution : maximum relative variation in \{X\}=\{$u, r, \ln
    f, T, L_r, P, \varrho$, $M_r$\}\\ between 2 adjacent shells.
\end{tabular}
\\ \hline
\HPT{dlnvmi} & $\displaystyle \Bigl\{\frac{\Delta T}{T}\Bigl\}_{\min}$ &
\hspace{-0.25cm}\begin{tabular}{l} Luminosity profile constrain if $\Delta
  T/T$ is larger than \HPL{dlnvmi}.
\end{tabular}
\\ \hline
\HPT{dlnenuc}$^\dag$ & $\triangle \varepsilon _{\rm nuc,\max }$ &
Equivalent to  
\HPL{dmrma}, but for $\varepsilon _{\rm nuc}$ specifically. \\ \hline
\HPT{dmrma} & $\triangle m_{\max }$ & Maximum increase of relative mass
allowed between adjacent shells. \\ \hline
\HPT{dmrmi} & $\triangle m_{\min }$ & Minimum increase of relative mass
allowed between adjacent shells. \\ \hline \hline

\multicolumn{3}{|c|}{\sf time-step} \\ \hline\hline
\HPT{dtin} & $\triangle t_{\rm in}$ & 
\begin{tabular}{l}
Initial evolution timestep of a model sequence. \\
(if zero, use previous evolution time step)
\end{tabular}
\\ \hline
\HPT{dtmin} & $\triangle t_{\min }$ & Minimum evolution timestep allowed.
\\ \hline
\HPT{dtmax} & $\triangle t_{\max }$ & Maximum evolution timestep allowed.
\\ \hline
\HPT{facdt} & $>1$ & 
\begin{tabular}{l}
Factor by which the evolution timestep is increased (in case of convergence)
\\ 
or reduced (by 2, in case of crash).
\end{tabular}
\\ \hline
\HPT{fkhdt}$^\dag$ & $\leq 1$ & 
\begin{tabular}{l}
Fraction of the Kelvin-Helmholtz timescale considered as maximum evolution
\\ 
timestep allowed, only for contracting phases (i.e. nphase = 1, 3, 5 and 7).
\end{tabular}
\\ \hline
\HPT{ishtest} & f, t & 
\begin{tabular}{l}
Next evolution timestep estimated by the dependent variable evolution rate
\\ 
in all the shells (f) or just in shells where nuclear burning occurs (t).
\end{tabular}
\\ \hline
\HPT{fts}$^\dag$ & $<1$ & 
\begin{tabular}{l}
Relative maximum change of dependent variable values allowed in each shell
\\ 
between two consecutive models (to estimate the evolution timestep).
\end{tabular}
\\ \hline
\HPT{ftsh}$^\dag$ & $<1$ & 
\begin{tabular}{l}
The evolution timestep is not allowed to be larger than \HPL{ftsh} times
\\ 
the nuclear timescale corresponding to H burning (where $\varepsilon _{\rm nuc}$
is maximum).\\
When {\sf nphase=7}, \HPL{ftsh} controls {Ne burning}.
\end{tabular}
\\ \hline

\newpage\hline \textbf{{variable}} & \textbf{values} &
meaning \\ \hline\hline \hline

\HPT{ftshe}$^\dag$ & $<1$ & Same as \HPL{ftsh}, but corresponding to He
burning. \\ \hline
\HPT{ftsc}$^\dag$ & $<1$ & Same as \HPL{ftsh}, but corresponding to C
burning. \\ \hline
{\HPT{ftst}$^\dag$} & $<1$ & - Controls the max. temperature
increase allowed between 2 consecutive models. \\ 
& $<1$ & - Controls the increase in nuclear luminosities associated with
the different burning modes (H,He and C).\\ \hline
\HPT{ftacc}$^\dag$ & $<1$ & Same as \HPL{ftsh}, but corresponding to the
accretion timescale (if \HPL{iaccr}$>$0). \\
& $<1$ & Also controls the flame speed (activated if \HPL{imodpr}=11). \\
\hline 

\multicolumn{3}{|c|}{\sf iterations} \\ \hline\hline
\HPT{itermin} & $>2$ & Minimum number of iterations to converge a model. \\ 
\hline
\HPT{itermax} & $<999$ & Maximum number of iterations to converge a model.
\\ \hline
\HPT{itermix} & $-99 \rightarrow 99$ & 
\begin{tabular}{l}
Maximum number of iterations over which mixing procedure is applied \\ 
at each iteration (if \HPL{mixopt} is true).\\
If \HPL{itermix}$<0$ initial abundances  restored after the \HPL{itermix}
iterations.
\end{tabular}
\\ \hline
\HPT{icrash} & \begin{tabular}{l} $<10$ \\ \smallskip $-10<$ \end{tabular}& 
\begin{tabular}{l}
In case of crash, the evolution timestep is reduced \HPL{icrash} times \\ 
and increased 9$-$\HPL{icrash} times. \\
Negative values of \HPL{icrash} force the time step to increase $|$\HPL{icrash}$|$ times \\ 
and then decreased 9$-|$\HPL{icrash}$|$ times.
\end{tabular}
\\ \hline
{\HPT{numeric}} & 1 $\rightarrow$ 4 & 1 : convergence not followed in
the surface layers where $T <$ \HPL{tmaxioHe}  \\
& & 2-3 : first order spatial derivatives ({\sf zi}=$\frac{1}{2}$, {\sf zj}=$\frac{1}{2}$) \\ 
& & 2-4 : $1/\kappa_i = {\sf zi}/\kappa_{i-\frac{1}{2}}+{\sf zj}/\kappa_{i+\frac{1}{2}}$ \\
& & other values : second order spatial derivatives and $\kappa_i =
\kappa_{i-\frac{1}{2}}^{{\sf zi}} \kappa_{i+\frac{1}{2}}^{{\sf zj}}$\smallskip \\
\hline\hline
{\HPT{icorr}} & f,i,m,h,a,b,t & \begin{tabular}{ll}
f & : do not activate parameter adjustments in case of crash \\
i & : f + increase tolerance on velocity \\
m & : f + mesh disabled after 1 crash \\
h & : f + deactivate acceleration routine after 2 crashes (iacc=f) \\
a & : f + constraint time-step during third dredge-up and set egrav and its \\
  & \hspace{0.2cm} derivatives at 0 in the atmosphere \\
b & : set egrav and its derivatives at 0 in the atmosphere \\
t & : constraint time-step during third dredge-up and check that pressure is a \\
  & \hspace{0.2cm} decreasing function of mass
\end{tabular} \\ \hline\hline

\multicolumn{3}{|c|}{\sf tolerances} \\ \hline\hline
\HPT{tol\_u} & $\epsilon \left( u\right) $ & Tolerance to be reached by the
dependent variable $u$ for the model to converge. \\ \hline 
\HPT{tol\_lnr} & $\epsilon \left( \ln r\right) $ & Same as \HPL{tol\_u}, but for
dependent variable $\ln r$. \\ \hline
\HPT{tol\_lnf} & $\epsilon \left( \ln f\right) $ & Same as \HPL{tol\_u}, but
for dependent variable $\ln f$. \\ \hline
\HPT{tol\_lnT} & $\epsilon \left( \ln T\right) $ & Same as \HPL{tol\_u}, but
for dependent variable $\ln T$. \\ \hline
\HPT{tol\_l} & $\epsilon \left( l\right) $ & Same as \HPL{tol\_u}, but for
dependent variable $l$. \\ \hline\hline

\multicolumn{3}{|c|}{\sf convergence} \\ \hline\hline
\HPT{phi} & $0 \le \varphi \le 1$ & 
\begin{tabular}{l}
Correction apply to the linear extrapolation to determine the initial guess\\
structure ($x^{n+1} = x^n + \HPL{phi} \times [x^n-x^{n-1}] \times
\Delta t^{n+1}/\Delta t^n$).
\end{tabular}
\\ \hline
\HPT{alpha} & $0\rightarrow 1$ & 
\begin{tabular}{l}
Fraction of the Newton-Raphson correction actually applied in all the shells
\\ 
for each dependent variable that is considered for the first four iterations.
\end{tabular}
\\ \hline
\HPT{iacc} & f, t & 
\begin{tabular}{l}
Allow or not the current \HPL{alpha} value to be modified for all the \\
dependent variables
every four iterations (after the first four ones).
\end{tabular}
\\ \hline
\HPT{alphmin} & $0\rightarrow 1$ & Minimum value allowed for \HPL{alpha}
(if \HPL{iacc} is true). \\ \hline
\HPT{alphmax} & $0\rightarrow 1$ & Maximum value allowed for \HPL{alpha}
(if \HPL{iacc} is true). \\ \hline
\HPT{sigma} & $0\le \sigma \le 1$ & Numerical scheme.
\begin{tabular}{l}
0 : explicit scheme \\ $0 < \sigma < 1$ semi-implicit \\ 1 : implicit scheme
\end{tabular}
\\ \hline
\HPT{vlconv} & f,t & Treatment of equation of transport.
\begin{tabular}{l}
f :  $\rm \partial \ln T/\partial \ln P= \nabla_{conv} \ or\ \nabla_{rad} $ \\
t :  $L = L_{\rm rad} + L_{\rm conv}$
\end{tabular}
\\ \hline
\end{longtable}
\end{center}
\vspace{-1cm}
$^\dag$ if set to zero, not accounted for \\
\sout{word} : parameter not used (i.e. should be removed) \\
\HPL{partialmix} : convective zone treated as radiative if $\tau_{\rm conv}
\gg \texttt{dtn}$ (see \HPL{nuclopt}) \\
\HPL{lmix} : solution accepted only if during the last 2 iterations the
convective boundaries have NOT changed (and the tolerance fulfilled) (see
\HPL{dtnmix}) 

\newpage

\begin{center}
{\Large \sf \textbf{VARIABLES NAMES}}
\renewcommand{\baselinestretch}{1.05}
\vspace{1.0cm}
%\large

\begin{tabular}[@{\renewcommand{\baselinestretch}{5}}]{|l| lr|}
%\begin{tabular}{|l| lr|}

\hline
{\bf \sf variable name} & {\bf \sf Centered variables shell [i+1/2] } &
{\bf \sf symbol }\\\hline \hline
rho & density & $\rho$ \\
T & temperature  &T \\
P & pressure  &P \\
eint & internal energy & $E_{\rm int}$ \\
abad & adiabatic gradient & $\nabla_{\rm ad}$ \\
cp & specific heat a constant P & $\rm c_P$ \\
enupla & plasma neutrino energy loss rate & $\varepsilon_\nu$ \\
enucl & nuclear energy production rate & $\varepsilon_{\rm nuc}$ \\
egrav & gravothermal energy production rate & $\varepsilon_{\rm grav}$ \\ 
khimu & $\chi_\mu =  \frac{\partial \ln P }{\partial \ln \mu}\big|_{\rho,T}$ &
$\chi_\mu$ \\
khirho & $\chi_\rho = \frac{\partial \ln P }{\partial \ln
  \rho}\big|_{\mu,T}$ &$\chi_\rho$  \\
khit & $\chi_T =  \frac{\partial \ln P }{\partial \ln T}\big|_{\mu,\rho}$
& $\chi_\rho$   \\
ksirho & $\xi_\rho = \frac{\partial \ln P }{\partial \ln
  \rho}\big|_{T}$ & $\xi_\rho$  \\
phiKS & $\varphi = \frac{\partial \ln \rho }{\partial \ln
  \mu}\big|_{P,T} = - \frac{\chi_\mu}{\zeta_\rho} = - \phi$  & $\varphi$ \\
deltaKS & $\delta = -\frac{\partial \ln \rho }{\partial \ln
 T}\big|_{P,\mu}= \frac{\zeta_T}{\zeta_\rho}$  & $\delta$ \\
ksiT & $\xi_T = \frac{\partial \ln P }{\partial \ln
  T}\big|_{\rho}$ & $\xi_T$  \\
zrho & $\zeta_\rho = \chi_{\rho}+ \chi_\mu \xi_\rho$ & $\zeta_\rho$ \\
zt & $\zeta_T = \chi_T+ \chi_\mu \xi_\rho$ & $\zeta_T$ \\
V\_circ & horizontal component of the meridional circulation & $V_{\rm circ}$ \\
xKt & thermal diffusivity : $\mathrm{Flux} = -K_T \rho c_P \displaystyle \frac{\partial 
  T}{\partial r} $ & $K_T$  \\
rhmoy & average density $\frac{1}{M} \int \varrho\, \mathrm{d} m$ &
$\varrho_m$  \\\hline 
\hline
 & {\bf \sf Edge/interface  variables shell [i]}  &  \\ \hline\hline
 m & mass  & $M_r$ \\
 u & velocity  & $u_r$ \\
 r & radius  & r \\
 lum & luminosity  & $L_r$\\
 gmr & gravitational field & $\displaystyle \frac{{\cal G} M_r}{r^2}$  \\
 hp & pressure scale height & $\rm H_P$ \\
 ht & temperature scale height & $\rm H_T$ \\
 crzc & shell type (radiative, conv., semi, thermohaline, ...) &  \\
 abrad & radiative gradient & $\nabla_{\rm rad}$  \\ 
 abla & effective temperature gradient & $\nabla_*$  \\
 abel & temperature gradient of the convective cell & $\nabla_{el}$  \\ \hline
 & {\sf rotation - mixing - diffusion variables [i]}  & \\ \hline
 omega & rotation rate  & $\Omega$ \\
 theta & derivative of rotational rate  & $\theta$ \\
 ur & vertical component of the meridional circulation  & $U_r$ \\
 lambda & integration variable   & $\Lambda$ \\
 aux & integration variable  & $\cal{A}$    \\ 
 grav & normalized gravitational field & $g$ \\
epsmoy & average nuclear energy production $\frac{1}{M} \int
\varepsilon_{\rm nuc} \mathrm{d} m$ & $\varepsilon_m$ \\ 
D & all diffusion coefficients : cd, Dconv, Dsc, Dherw, coefDtacho,
    Dhd, Dthc, & \\
 & xNt, xNu, xNr, Dhold, Dh  & D \\ 
viscosity & xnum, xnuvv, xnumol, xnurad & $\nu$ \\ \hline 
\end{tabular}

\end{center}
\printindex 

\end{document}

